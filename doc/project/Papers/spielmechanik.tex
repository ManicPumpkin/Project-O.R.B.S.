\section{Spielmechanik}
\label{sec:spielmechanik}
%	\begin{itemize}
%		\item 
%	\end{itemize}

\subsection{Spielmodi}
\label{subsec:spielmodi}
	\begin{itemize}
		\item Softcore (Spieler verliert nichts oder nur Gold)
		\item Mediumcore (Spieler verliert einen Teil seiner Ausrüstung, kann wieder aufgesammelt werden)
		\item Hardcore (Spieler verliert seine Ausrüstung beim Sterben, kann nicht mehr aufgesammelt werden)
		\item Deathcore (Tot ist TOT!)
	\end{itemize}


\subsection{Story}
\label{subsec:story}
Eine Story gibt es nicht. Es werden aber hier und da, den Spieler und das Universum betreffende Fragen gestellt.


\subsection{Charakter-Erstellung}
\label{subsec:charakter-erstellung}
	\begin{itemize}
		\item verschiedene Haare usw.
		\item Kleidung
		\item dick, dünn, groß, klein usw.
	\end{itemize}

\subsection{Singleplayer}
\label{subsec:singleplayer}
%	\begin{itemize}
%		\item 
%	\end{itemize}

\subsection{Multiplayer}
\label{subsec:multiplayer}
%	\begin{itemize}
%		\item 
%	\end{itemize}

\subsubsection{Verbindung der Singleplayer-Universen}
\label{subsubsec:verbindung_der_singleplayer_universen}
	\begin{itemize}
		\item Universen werden nebeneinander dargestellt
		\item es wird eine Direktverbindung von Start-Orb zu Start-Orb erstellt
		\item Spieler spielen aktiv im Universum des jeweils anderen mit
		\item d.h. sie können alle Orbs erkunden, die der andere Universen-Besitzer schon erkundet hat. Sie können Ressourcen im anderen Universum abbauen. Sie können aktiv bauen, Magie wirken, Teleporter bauen, Orbs erkunden usw.
	\end{itemize}

\subsubsection{\textit{Multiplayer-Modi}}
\label{subsubsec:multiplayer_modi}
	\begin{itemize}
		\item Capture the Flag
		\item King of the Hill
		\item Build'n Destroooooy (King Arthur's Gold)
		\item (Team-)Death-Match
		\item eventuell MMORPG-Server (mehrere Spieler spielen in einem Universum)
	\end{itemize}

%\subsection{Städte}
%\label{subsec:staedte}
%	\begin{itemize}
%		\item NPC-Städte mit Politiksystem (einhergehend mit Fraktionen)
%	\end{itemize}

%\subsection{Fraktionen}
%\label{subsec:Fraktionen}
%	\begin{itemize}
%		\item NPC-Fraktionen => Verbündete, Neutrale, Feinde
%		\item Der Spieler hat die Möglichkeit eine eigene Fraktion mit eigenen "Anhängern" zu gründen, die sich in der Welt behaupten muss
%		\item Loyalitätsskalen
%		\item Aggressorenskalen
%	\end{itemize}

%\subsection{Handel}
%\label{subsec:handel}
%	\begin{itemize}
%		\item Spieler kann Güter zum Tausch verwenden
%		\item Es kann allerdings auch eine Währung eingesetzt werden
%		\item Karawanen die regelmäßig für Gold sorgen, die allerdings auch beschützt werden müssen (Piraten, Räuber, Ronins usw.)
%	\end{itemize}

\subsection{Ressourcen}
\label{subsec:ressourcen}
%	\begin{itemize}
%		\item 
%	\end{itemize}

\subsection{Items}
\label{subsec:items}
%	\begin{itemize}
%		\item 
%	\end{itemize}

\subsection{Crafting-System}
\label{subsec:crafting-system}
Der Spieler kann selber craften, kann es aber von NPCs machen lassen:
\subsubsection{via Spieler}
\label{subsubsec:spieler-crafting}
	\begin{itemize}
		\item Rohstoffe sammeln
		\item z.B. Erze am Ofen schmelzen
		\item Komponenten erstellen (z.B. Griff aus Holz an der Drehbank; Klinge aus Metall am Amboss)
		\item Produkt (Schwert) aus Komponenten zusammensetzen (z.B. Werkbank)
	\end{itemize}

\subsubsection{via NPC}
\label{subsubsec:NPC-crafting}

	\begin{itemize}
		\item Rohstoffe sammeln kann durch Anweisungen auf NPCs übertragen werden 
		\subitem Der Weg zu den Rohstoffen muss gegeben sein
		\subitem Spieler wählt Rohstoff aus Prioritätenliste, der von den NPCs abgebaut werden soll
		\subitem NPCs bauen also NUR die Rohstoffe ab. Sie legen keine Wege zu den Rohstoffen an.
		\subitem somit bleibt das `Terraforming' komplett in den Händen des Spielers
		\item NPCs bringen Rohstoffe zurück in das Lagerhaus, wo die Rohstoffe für andere NPCs zur Verfügung stehen
		\item NPC Schmied nutzt das vorhandene Eisenerz aus dem Lagerhaus, um Barren herzustellen usw.
	\end{itemize}

\subsection{Magie-System}
\label{subsec:magie-system}
	\begin{itemize}
		\item da Magie übermächtig ist, muss man sie sich verdienen (lernen, studieren, forschen usw.)
		\item Spieler besitzt ein Zauberbuch (kein Item, sondern eher ein Menü), indem viele verschieden Zauber zur Verfügung stehen (müssen aber erst irgendwie gelernt werden)
		\subitem Zauber können aus dem Zauberbuch in die Schnellleiste gezogen und danach gewirkt werden
		\item Jeder Zauber kann auf verschieden Art und Weise gewirkt werden (ähnlich wie bei Magicka $\rightarrow$ selbes Prinzip bei anderen Kämpfern z.B. Angriff, Blocken, usw.; siehe Kampfsteuerung \ref{subsec:kamp-system})
		\item Items ändern nur die Stati des Spielers.
		\item Es wird keine Elemente und Kombinationen aus den Elementen geben (zu viel des Guten)
	\end{itemize}

\subsection{Kampf-System}
\label{subsec:kamp-system}
Verschieden Art und Weisen zu Kämpfen:
	\begin{itemize}
		\item Nahkampf
		\item Fernkampf
		\item Zauber
		\item Alchemie
	\end{itemize}

\paragraph{Kampfsteuerung\\}
Es gibt 2 Angriffsarten. LMB \& RMB. Man kann zusätzlich Blocken Space.
\subparagraph{Nahkämpfer}
\begin{itemize}
\item LMB Zustechen
\item RMB schwerer Hieb
\item Space Parrieren
\end{itemize}

\subparagraph{Fernkämpfer}
\begin{itemize}
\item LMB Schnellschuss (braucht nicht so viel Spannung)
\item RMB Pierce-Schuss (geht durch Gegner durch; Teilt jedem Schaden aus)
\item Space Dash (Gewisse Distanz schnell überwinden ohne Schaden zu erleiden)
\end{itemize}

\subparagraph{Zauberer}
\begin{itemize}
\item LMB Angriff 1
\item RMB Angriff 2
\item Space Barriere bzw. Schutzzauber
\end{itemize}

\subparagraph{Alchemist}
\begin{itemize}
\item LMB schmeißt Trank (Effekt, je nach Trank)
\item RMB beschwört Wesen, je nach Trank
\item Space beschwört Wesen für die Dauer des Tastendrucks direkt vor dem Spieler. Das Wesen steckt den Schaden ein.
\end{itemize}